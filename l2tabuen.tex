%% This is the English translation by Juergen Fenn
%% <juergen.fenn@gmx.de>, (c) 2018, of version 2.4 of l2tabu.tex by
%% Mark Trettin <Mark.Trettin@gmx.de> and Marc Ensenbach
%% <Marc.Ensenbach@post.rwth-aachen.de>. The original document was
%% first published in 2005. It was last modified on 3 Feb 2016. It can
%% be found on CTAN at <CTAN:info/l2tabu/german>.
%%
%% Project URL: <https://github.com/schneeschmelze/l2tabuen>
%% CTAN home: <CTAN:info/l2tabu/english>.
%%
%% The original file l2tabu and this English translation may be
%% distributed only subject to the terms and conditions set forth in
%% the Open Publication License, v1.0 or later (the latest version is
%% presently available at <http://www.opencontent.org/openpub/>).

\documentclass[abstract=on, singlepage=on, paper=a4]{scrartcl}

\usepackage[utf8]{inputenc}
\usepackage[T1]{fontenc}
\usepackage{mathpazo}
\usepackage[scaled=.90]{helvet}
\linespread{1.05}%
\usepackage{textcomp}
\sloppy % yes, I know---it's crude, but it works

\usepackage[ngerman, english]{babel}
\usepackage[style=verbose-ibid, backend=biber, language=english]{biblatex}
\usepackage[english]{csquotes}
\bibliography{l2tabuen}

\usepackage{url}

\newcommand{\Author}[1]{\emph{#1}}
\newcommand{\Command}[1]{\texttt{\textbackslash{}#1}}
\newcommand{\Doku}[1]{\emph{#1}}
\newcommand{\File}[1]{\texttt{#1}}
\newcommand{\Legal}[1]{\emph{#1}}
\newcommand{\Option}[1]{\texttt{#1}}
\newcommand{\Package}[1]{\textsf{#1}}
\newcommand{\Programme}[1]{\textsf{#1}}

\title{An essential guide to \LaTeXe{} usage}
\author{German version\\
  by Mark Trettin\\and Marc Ensenbach
  \and{}
  English translation\\
  by Jürgen Fenn
}
\date{\today}

\begin{document}
\pagestyle{headings}
\maketitle
\begin{abstract}
  \noindent This is the English translation of version~2.4 of
  \Doku{l2tabu}. It is a complete rewrite of the last English edition
  published over ten years ago. This paper goes back to a discussion
  thread on the German \TeX{} newsgroup \texttt{de.comp.text.tex}
  about bad habits of \LaTeX{} users that seem to die hard. We try to
  give a demonstration of the most common mistakes we have come across
  over time, and we give some advice on how to improve on your
  \LaTeX{} usage. No attempt, however, is made to compete with proper
  introductions such as \Doku{l2short}, or the \Doku{UK \TeX{}
    FAQ}. Of course, suggestions and comments are always welcome.
\end{abstract}

\subsection*{Legal notice}
\label{sec:legal-notice}

Copyright \copyright{} 2003--2007 by \Author{Mark Trettin}, 2009--2016
by \Author{Marc Ensenbach}, for the English translation 2005--2018 by
\Author{Jürgen Fenn}.

\medskip\noindent
This material may be distributed only subject to the terms
and conditions set forth in the \Legal{Open Publication License}, v1.0
Or Later (the latest version is presently available at
\url{http://www.opencontent.org/openpub/}).

\subsection*{Acknowledgements}
\label{sec:acknowledgements}

Thanks to Ralf Angeli, Christoph Bier, Christian Faulhammer, Jürgen
Fenn, Ulrike Fischer, Ralf Heckmann, Yvon Henel, Yvonne Hoffmüller,
David Kastrup, Markus Kohm, Thomas Lotze, Gonzalo Medina Arellano,
Frank Mittelbach, Heiko Oberdiek, Walter Schmidt, Uwe Siart, Axel
Sommerfeld, Stefan Stoll, Knut Wenzig, Emanuele Zannarini, and
Reinhard Zierke for advice and corrections. 

\subsection*{Contributors to the English version}
\label{sec:contr-engl-vers}

Barbara Beeton, Karl Berry, Christoph Bier, Stephen Eglen, Klas
Elmgren, Gernot Hassenpflug, Yvon Henel, Hendrik Maryns, Walter
Schmidt, Maarten Sneep, Stefan Ulrich, José Carlos Santos, Knut
Wenzig, Bruno Wöhrer, and Federico Zenith have contributed to the
English version, making suggestions, or simply encouraging
development.

\medskip\noindent
In case we forgot someone, please send us an email.

\subsection*{Translator's note}
\label{sec:translators-note}

I have attempted to keep as true to the original version of
\Doku{l2tabu} as possible. However, there are some points that need
some more explanations, as it turns out that most of us do not have
easy access to usenet any more. So, for easy reading I have looked up
the references to newsgroup postings and explained in a footnote what
the source really said. I have marked these footnotes as translator's
notes to make it clear that I added something to the text.
% I also have
% replaced the references to the new \Doku{German \TeX{} FAQ} which used
% to be hosted at DANTE's website by those at \url{texfragen.de} which
% is the succeeding project at the time of writing this
% translation.
~--- \Author{J.\,F.}

\subsection*{How to get in touch with the authors}
\label{sec:how-get-touch}

You can get in touch with us by email: Mark Trettin~--
\url{mark.trettin@gmx.de}, Marc Ensenbach~--
\url{marc.ensenbach@post.rwth-aachen.de}. All comments on the English
version should please be addressed to Jürgen Fenn~--
\url{juergen.fenn@gmx.de}.

\newpage
\tableofcontents

\section{\enquote{Deadly sins}, or the most severe mistakes in using \LaTeXe}
\label{sec:deadly-sins}

These are the most severe mistakes tha kept turning up in
\texttt{de.comp.text.tex} at the time the paper was first drafted.

\subsection{\Package{a4.sty}, \Package{a4wide.sty}}
\label{sec:a4.sty-a4wide.sty}

These \enquote{two} packages should not be used any more. They should
be replaced by class options \Option{a4paper}, or \Option{paper=a4} in
KOMA-Script from version~3. There are two reasons for this. On the one
hand, these packages produced a rather bad layout. On the other, it
turned out that there have been several versions of these packages
around. So, even today there still is no guarantee that you will get
the same layout on different machines.

\subsection{How to adjust page layout}
\label{sec:adjust-page-layout}

The page margins produced by the \LaTeX{} standard classes
(\File{article.cls}, \File{report.cls}, \File{book.cls}) printed on A4
paper look too wide in the eyes of most Europeans. You might like to
use the classes from the KOMA-Script bundle instead
(\File{scrartcl.cls}, \File{scrreprt.cls},
\File{scrbook.cls}). KOMA-Script also provides \Package{typearea.sty}
which allows for adjusting page layout according to rules set out by
the German typographer \Author{Jan Tschichold}. See the KOMA-Script
guide for details.\footcite{kohm_guide_2018}

Some users change parameters like \Command{oddsidemargin}, or they
employ the package \Package{anysize.sty}. However, this is not a good
idea, either, because the latter package is obsolete. Package
\Package{vmargin} is another case in point, as it also shows some bad
side effects.\footnote{Translator's note: As \Author{Uwe Siart}
  pointed out in \protect\url{<uirrzdr16.fsf@uwe-siart.de>} on
  1~February 2006, \Package{vmargin} does not comply with the
  \TeX{}book. As \LaTeX{} itself keeps to this standard for
  \Command{hoffset} and \Command{voffset}, \Package{vmargin} is
  incompatible with both \LaTeX{} and most other \LaTeX{} packages
  that are standard compliant.}

A valid option for adjusting page margins, however, is
\Package{geometry.sty}. Please be careful not to touch
\Command{hoffset} and \Command{voffset}, unless you are absolutely
sure about what you are doing.

\subsection{How to edit packages and classes}
\label{sec:edit-pack-class}

Never ever edit document classes (\File{article.cls},
\File{scrbook.cls}, etc.) of packages (\File{varioref.sty},
\File{color.sty}) directly! You might like to use a \enquote{container
  class}, or a \enquote{container package} instead. You also might
like to move the class or the style files to a different file before
editing them. See the \Doku{German \TeX{} FAQ} for more
details.\footcite{voss_wie_2009}

\medskip\noindent\textbf{Note:} Any additional classes or packages
should be installed either in your local \File{texmf} tree for all
users, or in you personal \File{texmf} tree in your home directory.
If you need a file Files for one project only you may keep it in the
project directory.\footnote{Translator's note: Make sure to run
  \Programme{mktexlsr} after installing any additional files to your
  \TeX{} distribution in order to update its file name database.}

\subsection{How to control inter-line space with
  \Command{baselinestretch}}
\label{sec:control-inter-line}

As a rule, all parameters in a document should be edited on the
highest level in the user interface. Thus, inter-line space can by
controlled in three ways:
\begin{enumerate}
\item With \Package{setspace.sty}.
\item Using the \LaTeX{} command \Command{linespread\{<factor>\}}.
\item Redefining \Command{baselinestretch}.
\end{enumerate}

Redefining parameters such as \Command{baselinestretch} takes place on
the lowest level. This is why it should only be done in \LaTeX{}
packages. For users there is a command called
\Command{linespread\{<factor>\}} which serves exactly this
purpose. However, the best way to deal with inter-line space is to use
\Package{setspace.sty} because it also deals with inter-line space in
footnotes and list environments which in most cases should remain
unchanged.

Use \Package{setspace.sty} for setting inter-line space to one-half or
to double spacing, while minor adjustments that fit fonts other than
Computer Modern may be done with a \Command{linespread\{<factor>\}}
command. E.g., the font Palatino fits \Command{linespread\{1.05\}}.

\subsection{How to control paragraph indent and the spread between
  paragraphs}
\label{sec:control-paragr}

Sometimes it makes sense to adjust paragraph indent
(\Command{parindent}). If you do so, please note:
\begin{itemize}
\item Parindent should be measured according to font size (em), i.e.,
  not in absolute terms (mm). What's more, the unit 1\,em also depends
  on the font face employed, as every font designer adjusts this
  measure to the individual font.
\item \LaTeX{} syntax should be preferred over \TeX{} code. E.g., it
  is simpler to parse \LaTeX{} code with external programmes or
  scripts, it is easier to maintain, and it avoids incompatibilities
  with other packages (\Package{calc.sty}).
\end{itemize}

\printbibliography

\end{document}
