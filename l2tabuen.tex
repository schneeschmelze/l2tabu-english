%%%------------------------------------------------------------------
%%% Filename:    l2tabuen.tex
%%% Author:      Mark Trettin, J�rgen Fenn <juergen.fenn@gmx.de>
%%% Created:     16 Juni 2007
%%% Time-stamp:  none
%%% Version:    $Id: l2tabuen.tex,v 1.8.5.7 $
%%%
%%% Copyright (C) 2004--2007 by Mark Trettin and J�rgen Fenn
%%%------------------------------------------------------------------
%%% TODO:
%%% - \input{../asdf/} instead of \input ../asdf/

\documentclass[11pt,a4paper,pagesize,tablecaptionabove,abstracton,pointlessnumbers]{scrartcl}
%--------------- Basic Configuration (German, 8bit) ----------------
%\usepackage[ngerman]{babel}     % Switch to German
\usepackage[latin1]{inputenc}   % Direct input of German umlauts etc.
\usepackage[T1]{fontenc}        % T1 Fonts
%%% Times/Helvetica/Courier *only* for minimizing the size of this file
%% as these fonts do not have to be inserted into the output document.
\usepackage{mathptmx}           % Times/Maths \rmdefault
\usepackage[scaled=.90]{helvet} % Scaling Helvetica \sfdefault
\usepackage{courier}            % Courier \ttdefault
\usepackage{multicol}

\typearea[current]{current}     % Calculate typearea anew
%------------ additional packages (graphics, tables) ---------------
\usepackage{xspace}             % automatic blanks after macros
\usepackage{booktabs,array}     % nicer tables
\usepackage[english]{varioref}  % modifyed for the version publicised
                                % as current versionf of fancy, and
                                % varioref do not work together
                                % (still not?!)
\usepackage{textcomp}           % additional characters
\usepackage[nofancy]{rcsinfo}   % RCS Informationen in title
\usepackage{enumerate}
\usepackage{calc}               % we must calculate
\usepackage{eurosans}           % the Euro
%--------------------------- Formatting ---------------------------
%%% Captions
\setkomafont{caption}{\normalcolor\small\sffamily\slshape}
\setkomafont{captionlabel}{\normalcolor\upshape\small\sffamily\bfseries}
%--------------------------- PDF Options ---------------------------
\ifpdfoutput{%                 % needs KOMA-Script class
  \usepackage{hyperref}        % hyperref settings for pdf
  \hypersetup{%
    pdfauthor={(c) 2007 Mark Trettin and Juergen Fenn},%
    pdftitle={l2tabuen -- An essential guide to \LaTeXe\ usage},%
    pdfsubject={Some hints on LaTeX2e},%
    bookmarksopen=true,% Temp
    backref=true,%
    pdffitwindow=true,%
    pdfpagelayout=OneColumn,%
    colorlinks=true,%
    linkcolor=red,%
    linktocpage=true,%
    backref=true,%
    pdfstartview=FitH,%
    bookmarksopen=true,%
    bookmarksopenlevel=2,%
    bookmarksnumbered=false,%
    urlcolor=blue%
  }%
  \usepackage[activate=normal]{pdfcprot} % character protruding using pdftex
  \newcommand{\EMail}[1]{\href{mailto:#1?Subject=[l2tabu.pdf]}{email: \texttt{#1}}}
  \newcommand{\News}[1]{\href{news:#1}{\texttt{#1}}}
  \newcommand{\MID}[2]{\href{http://groups.google.com/groups?as_umsgid=#1}%
    {Message-ID: \texttt{<#2>}}}
  % \usepackage{color}           % For preview-latex (in pdf tree)
  }%
%--------------------------- PS Options ----------------------------
  {%
    \usepackage{hyperref}
    \newcommand{\EMail}[1]{\href{mailto:#1}{Email: \texttt{#1}}}
    \newcommand{\News}[1]{\href{news:#1}{\texttt{#1}}}
    \newcommand{\MID}[2]{\href{M-ID: <#1>}{Message-ID: \texttt{<#2>}}}
  }
\usepackage{color} % comment out for preview-latex
%-------------------------- new commands --------------------------
%%% documentation: SansSerif
\newcommand{\TB}{\textbackslash}
\newcommand{\Doku}[1]{\textsf{#1}\xspace}
%%% Paket ,Klasse, Bibstil (packages, classes, bib style): SansSerif, Slanted
%\newcommand{\Paket}[2][sty]{\textsf{\textsl{#2.#1}}\xspace}
\newcommand{\Paket}[1]{\textsf{\textsl{#1.sty}}\xspace}
\newcommand{\Klasse}[1]{\textsf{\textsl{#1.cls}}\xspace}
\newcommand{\Bst}[1]{\textsf{\textsl{#1.bst}}\xspace}
%%% Options: SansSerif
\newcommand{\Option}[1]{\textsf{#1}\xspace}
%%% \Use{graphicx}         --> \usepackage{graphicx}
%%% \UseO{dvips}{graphicx} --> \usepackage[dvips]{graphicx}
\newcommand{\Use}[1]{\texttt{\textbackslash usepackage\{#1\}}}
\newcommand{\UseO}[2]{\texttt{\TB usepackage[#1]\{#2\}}}
\newcommand{\UseV}[2]{\texttt{\TB usepackage\{#1\}[#2]}}
\newcommand{\UseOV}[3]{\texttt{\TB usepackage[#1]\{#2\}[#3]}}
%%% Footnotes
\deffootnote{2.5em}{1em}{\thefootnotemark\ \ \ }%
\setlength{\footnotesep}{12pt}
%%% Nonfrenchspacing
\nonfrenchspacing
%%% Bibliography
\newcommand{\Bib}[1]{\texttt{\textbackslash bibliographystyle\{#1\}}}
%%% environment TODO: redefine, for this looks rather bad
\newcommand{\Env}[2]{\raggedright\texttt{\textbackslash begin\{#1\}\\
    #2\\ \textbackslash end\{#1\}}}
%%% \newcommand
\newcommand{\NewCom}[3][]{\texttt{\textbackslash newcommand{#1}\{\textbackslash#2\}\{#3\}}}
\newcommand{\ReNewCom}[3][]{\texttt{\textbackslash renewcommand{#1}\{\textbackslash#2\}\{#3\}}}
%%% Stolen from  scrguide ;-)
\DeclareRobustCommand*{\Macro}[1]{\mbox{\texttt{\char`\\#1}}}
\DeclareRobustCommand*{\LMacro}[2]{\mbox{\texttt{\char`\\#1\{#2\}}}}
\DeclareRobustCommand*{\GMacro}[2]{\mbox{\texttt{\{\char`\\#1\ #2\}}}}
% DANTE
\providecommand{\CTANserver}{ftp.dante.de}
% Make CTAN:// be an alias of ftp://\CTANserver/tex-archive/
\makeatletter
\def\url@#1{\expandafter\url@@#1\@nil}
\def\url@@#1://#2\@nil{%
  \def\@tempa{#1}\def\@tempb{CTAN}\ifx\@tempa\@tempb
    \hyper@linkurl{\Hurl{#1:#2}}{ftp://\CTANserver/tex-archive/#2}%
  \else
    \hyper@linkurl{\Hurl{#1://#2}}{#1://#2}%
  \fi
}
\makeatother
%%% Farben
\definecolor{gruen}{rgb}{0,0.55,0}
\definecolor{rot}{rgb}{1,0,0}
\newcommand{\FIXME}[1]{\marginline{FIXME: #1!}}
\newsavebox{\Quelle}

\newenvironment{bspcode}[1]{%
  \sbox{\Quelle}{\footnotesize From: #1}
  \center
  \rule{\linewidth}{1.5pt}
}{%
  \rule{\linewidth}{1.5pt}
  \usebox{\Quelle}
  \endcenter%
}

%%% Macros for replacing FIXME:
\newcommand{\Ersetze}[2]{\par\noindent Replace: \textcolor{red}{#1}
  by \textcolor{gruen}{#2}}
\newcommand{\Ersetzx}[3][.5\textwidth]{%
  \par\noindent%
  \begin{minipage}[t]{#1}
    \raggedright
    Replace:\\
    \textcolor{rot}{#2}
  \end{minipage}%
  \hfill%
  \begin{minipage}[t]{(\linewidth - #1)-.02\linewidth}
    \raggedright
    by\\
    \textcolor{gruen}{#3}
  \end{minipage}%
}
% Width for  "Ersetze" macros :-(
\newsavebox{\Breite}
%%% A hack by Heiko Oberdiek for acroread under Linux
\pdfstringdefDisableCommands{%
  \edef\quotedblbase{\string"}%
  \edef\textquotedblleft{\string"}%
}
%%% a very, very crude Hack for typesetting a URL in one line
%%% *without* any additional spacing. :-(
\newcommand{\biburl}[1]{\hfill\\URL:~\url{#1}}
%------------------- defining columns in tables --------------------
% needs array
\newcolumntype{v}[1]{>{\raggedright\hspace{0pt}}p{#1}} % v column, like
                                                       % p column, but
                                                       % raggedright
\newcolumntype{V}[1]{>{\footnotesize\raggedright\hspace{0pt}}p{#1}}
\newcolumntype{N}{>{\footnotesize}l}
\newcolumntype{C}{>{\footnotesize}c}
%----------------------------- title ------------------------------
\title{An essential guide to \LaTeXe{} usage\\[1em]
  \LARGE\mdseries\rmfamily Obsolete commands and packages}

\author{Original German version\thanks{Based on the German version
    1.8 of \Doku{l2tabu}.}\\
  by Mark Trettin\thanks{\EMail{Mark.Trettin@gmx.de}}%
  \and%
  English translation\\
  by J�rgen Fenn\thanks{\EMail{juergen.fenn@gmx.de}}}

\date{\today}

% \rcsInfoRevision\ \\
%   \rcsInfoLongDate}
%----------------------- begin of document -----------------------
\begin{document}
%\begin{Form}\end{Form}
\pagestyle{headings}
%%% RCS Info for version control
\rcsInfo $Id: l2tabuen.tex,v 1.8.5.7 2007/06/17 00:00:00 juergen Exp $
\maketitle
\begin{abstract}
  \noindent
  This is the English version \rcsInfoRevision{} of \Doku{l2tabu},
  focusing on obsolete commands and packages, and demonstrating the
  most severe mistakes most \LaTeX{} users are prone to make.  You
  should read this guide if you want to improve on your \LaTeX{} code.
\end{abstract}

\subsection*{Legal notice}
\label{sec:legalnotice}

Copyright \copyright{} 2007 by Mark Trettin and J�rgen Fenn. 
\medskip

Permission is granted to copy, distribute and/or modify this document
under the terms of the GNU Free Documentation License, Version 1.2 or
any later version published by the Free Software Foundation.
\emph{There are no invariant sections in this document.} Please
contact the translator of this version before distributing a modified
version of the following text. A copy of the licence is included in
appendix \ref{sec:fdl}.

\subsection*{Acknowledgements}%\addcontentsline{toc}{section}{Acknowledgements}%

Reading the German-language \TeX{} newsgroup \News{de.comp.text.tex}
one of us (Mark Trettin) found that most discussions were about
obsolete or, say, `bad' packages, and commands. So he decided to write
a brief summary to supply a practical guide to \LaTeX.  His paper was
called \Doku{altepakete.pdf} in the first place and soon it was
praised by senior developers writing to the group.  It is recommended
for reading ever since.  Later it was renamed by vote of participants
in \News{de.comp.text.tex} to \Doku{l2tabu}, corresponding to
\Doku{l2kurz}, the German title of \Doku{lshort}\cite{l2kurz:99}, and
the German translation of `taboo'.  This was about two years
ago.\footnote{\Doku{altepakete} was first announced on 18 February
  2003 on \News{de.comp.text.tex}.} I (J�rgen Fenn) joined Mark later
for translating his paper into English in order to help it spread to
those users who do not speak German.

In this article we give a demonstration of the most common mistakes in
using \LaTeX.  We also explain how to avoid them. This overview is
neither meant to replace introductions such as
\Doku{lshort}~\cite{l2kurz:99} nor the \Doku{De-TeX-FAQ}~\cite[version
72]{faq:02} nor the \Doku{UK FAQ}~\cite[version 3.16]{ukfaq:99}. Our
goal is just to give a small overview of how to write `good' \LaTeXe{}
code.

\subsection*{More translations of this paper}

Please note that besides the German original `Das
\LaTeXe-S�ndenregister oder Veraltete Befehle, Pakete und andere
Fehler. Tipps zu \LaTeXe' and this English version, there are more
translations of this paper.  They all can be found in the respective
subdirectories at
\begin{quote}
\url{CTAN://info/l2tabu/}
\end{quote}
So far \Doku{l2tabu} has been translated into English, French, and
Italian.

\subsection*{How to get in touch with the authors}

We are grateful for any suggestions, improvements, or comments.
Please address your emails directly to the translator of the
English\footnote{\EMail{juergen.fenn@gmx.de}. -- Download of
  \Doku{l2tabuen} from: \url{CTAN://info/l2tabu/english/}}, the
French\footnote{See the French translation \Doku{l2tabufr} by Yvon
  Henel at \url{CTAN://info/l2tabu/french/}}, or the
Italian\footnote{See the Italian translation \Doku{l2tabuit} by
  Emanuele Zannarini at \url{CTAN://info/l2tabu/italian/}} version
respectively.

Please tell us whether you have found \Doku{l2tabu} useful. We rely on
your feedback for improving our guide.

\subsection*{Thanks to\dots}

\dots{}
Ralf Angeli,
Christoph Bier,
Christian Faulhammer,
J�rgen Fenn,
Ulrike Fischer,
Yvon Henel, 
Yvonne Hoffm�ller,
David Kastrup,
Markus Kohm,
Thomas Lotze,
Frank Mittelbach,
Heiko Oberdiek,
Walter Schmidt,
Stefan Stoll,
Knut Wenzig,
Emanuele Zannarini, 
and
Reinhard Zierke
for tips, remarks, and corrections of the German original version.

\subsection*{Contributors to the English translation}

Barbara Beeton,
Karl Berry,
Christoph Bier, 
Stephen Eglen,
Klas Elmgren, 
Gernot Hassenpflug,
Yvon Henel,
Hendrik Maryns,
Walter Schmidt,
Maarten Sneep,
Stefan Ulrich,
Jos\'{e} Carlos Santos,
Knut Wenzig, 
Bruno W�hrer, and
Federico Zenith
have contributed to the English version, making suggestions, or
encouraging development.\\ 

\noindent If we have forgotten anyone please
send an email to the maintainer of the respective language version.

%\newpage


\clearpage
\tableofcontents 
%\clearpage

\section{`Deadly sins'\,---\,The most severe mistakes in using \LaTeXe}
\label{sec:todsunden}
In this section we probably have gathered together the most severe
mistakes that appear again and again in \News{de.comp.text.tex},
leaving regulars either with a flush of anger or weeping with tears
in their eyes.  \texttt{;-)}

\subsection{\Paket{a4}, \Paket{a4wide}}
\label{sec:paketa4-paketa4wide}

Do not use these `two' packages any longer. You should delete them
without replacement from your \LaTeX{} source. Use the class option
\Option{a4paper} instead. Speaking in terms of typography these
packages, or others similar to these do not provide good layout. What
is even worse, there is more than one version of these packages
around, and different versions of those packages are incompatible with
one another, providing deviating settings for page margins.  So you
may not trust that your document will look the same\,---\,or just as bad?
-- when being compiled on someone else's system when exchanging
\LaTeX{} source.

\Ersetze{\Paket{a4}, \normalcolor or \color{red} \Paket{a4wide}}{class option \Option{a4paper}}

\subsection{Modifying page layout}
\label{sec:layoutanderungen}
Page margins produced by the standard classes (\Klasse{article},
\Klasse{report}, \Klasse{book}) are often deemed too wide by European
users printing on A4 paper. They should use the corresponding classes
from the \KOMAScript{} bundle instead (\Klasse{scrartcl},
\Klasse{scrreprt}, \Klasse{scrbook}). These classes have been made
with a European point of view on typography in mind. You can also use
\Paket{typearea} which is part of \KOMAScript{} with any other
document class. The documentation included in the bundle provides some
more information.  Indeed, this very paper was typeset using
\Klasse{scrartcl}.

If you really need to use page margins altogether different from the
ones produced by \Paket{typearea} use \Paket{geometry}, or
\Paket{vmargin} because these packages provide reasonable proportions
in setting page margins. Do not use \Macro{oddsidemargin} or similar
commands for modifying page layout.

Under no circumstances change \Macro{hoffset}, or \Macro{voffset},
unless you really understand what \TeX{} is doing here.

\subsection{Changing packages and document classes}
\label{sec:ander-von-paket}
Never modify \LaTeX{} class files (\emph{e.g.}, \Klasse{article},
\Klasse{scrbook}) or packages (style files, \emph{e.g.},
\Paket{varioref}, \Paket{color}) directly! If you do not want to make
yourself a `container class', or a \texttt{.sty} file of your own you
should \emph{copy} the class, or style files, edit \emph{the copy},
and save it as a \emph{different} file using a \emph{different} file
name.

On how to create container classes see the
\Doku{De-TeX-FAQ}~\cite[question 5.1.5]{faq:02}.

\paragraph{Note:}
\label{sec:hinweis}

Install any additional files, or packages in the local texmf tree in
your \texttt{\$HOME} directory.  Otherwise these changes will be
overwritten when upgrading your \TeX{} distribution. Styles or
packages you only need in one particular project or which you may want
to hand on to someone you wish to share your project with may as well
be saved in the current working directory. See the
\Doku{De-TeX-FAQ}~\cite[question 5.1.4]{faq:02}, or the \Doku{UK
  FAQ}\cite[`Installing \LaTeX{} files', section K, `Where to put new
files', question 90]{ukfaq:99}.

\subsection{Changing inter-line space using \texttt{\textbackslash
    baselinestretch}} 
\label{sec:ander-des-zeil}

As a rule of thumb, parameters should be set on the highest possible
level within a user interface. So if you want to reset inter-line
space you can do so on three levels:
\begin{enumerate}
\item Either by using the \Paket{setspace} package;
\item or by using the \LaTeX{} command \LMacro{linespread}{<factor>};
\item or by redefining \Macro{baselinestretch}.
\end{enumerate}
Redefining parameters such as \Macro{baselinestretch} works on the
lowest \LaTeX{} level available\,---\,which should better be left to
packages. The \Macro{linespread} command is provided for this, so it
is a better way to get more inter-line space than fiddling with
\Macro{baselinestretch}. It is even better, though, to use
\Paket{setspace} which also takes care of space in footnotes and list
environments that you usually don't want to change when modifying
inter-line space.

So if you just need some more spacing between lines, say, you would
like to set spacing to one half or to double spacing, \Paket{setspace}
provides the easiest way to achieve this. However, if you only want to
use fonts other than Computer Modern you may use
\LMacro{linespread}{<factor>}. For example, when using Palatino
\LMacro{linespread}{1.05} would be appropriate.

\subsection[Parindent and the spread between paragraphs
(\texttt{\textbackslash{}parindent},
\texttt{\textbackslash{}parskip})]{Paragraph indent and the spread
  between paragraphs (\texttt{\textbackslash{}parindent},
  \texttt{\textbackslash{}parskip})}
\label{sec:absatz-und-abst}

It may make sense to change the indent of the first line in paragraphs
(\Macro{parindent}). However, if you do so, please note the following:
\begin{itemize}
\item Never use absolute sizes (\emph{e.g.}, `mm') to modify paragraph
  indent.  Use sizes that depend on font size, such as `em', for
  example.  The latter does \emph{not} mean that indent adapts
  automatically when changing the font size. Rather, the value that
  goes with the font currently activated is used.
\item Always use \LaTeX{} commands. For example, this may make it
  easier to parse\footnote{That is to say, analyse syntactically, or
    split up.}  a \LaTeX{} file through an external program, or
  script. Your code will be easier to maintain, too. So problems
  concerning compatibility with other packages can be avoided as well
  (\Paket{calc}, for example).
  \Ersetze{\Macro{parindent}\texttt{=1em}}{\LMacro{setlength}{\Macro{parindent}}\{1em\}}
\end{itemize}
%
In case you prefer some additional space between paragraphs to
paragraph indent for marking the start of a new paragraph (`zero
paragraph indent') do \emph{not} use 
\color{red}
\begin{verbatim}
\setlength{\parindent}{0pt}
\setlength{\parskip}{\baselineskip}
\end{verbatim}
\normalcolor
%
\Macro{parskip} should not be used as it will also modify settings for
list environments, table of contents, etc., and headings.

The \Paket{parskip} package, however, as well as the \KOMAScript\
classes go to some lengths to avoid these side effects. On how to use
these \KOMAScript{} class options (\Option{parskip},
\Option{halfparskip}, etc.)\  see \Doku{scrguien}~\cite{kohm:03}. When
using one of the \KOMAScript{} classes you do \emph{not} need to load
\Paket{parskip}.

\subsection{Separating maths formulae from continuous text using \texttt{\$\$\dots\$\$}}
\label{sec:abges-form-mit}

Please don't do this! \texttt{\$\$\dots\$\$} is a Plain \TeX{}
command.  It will modify vertical spacing within formulae, rendering
them inconsistent. This is why it should be avoided in \LaTeX{} (see
section~\vref{sec:mathematiksatz}; note the warning concerning
\texttt{displaymath} along with the \Paket{amsmath} package). What's
more, class option \Option{fleqn} won't work any more.

\Ersetze{\texttt{\$\$\dots\$\$}}{%
\parbox[t]{.3\textwidth}{%
\Macro{[}\texttt{\dots}\Macro{]}\\
\textcolor{black}{or}\\
\Env{displaymath}{\dots}}}

\subsection{\texttt{\textbackslash def} vs. \texttt{\textbackslash newcommand}}
\label{sec:def-vs.-newcommand}

\emph{Always} use \NewCom{<name>}{\dots} for defining
macros.\footnote{See \cite[section 2.7.2]{clsguide}, \cite[section
  3.4]{usrguide}.} 

\emph{Never} use \Macro{def}\LMacro{<name>}{\dots}. The main problem
with \Macro{def} is that no check is done on whether there already
exists another macro of the same name. So a macro defined earlier may
be overwritten without any error warning.

Macros may be re-defined using \ReNewCom{<name>}{\dots}.

If you know \emph{why} you need to use \Macro{def} you will probably
know about the pros and cons of this command. Then, you may as well
ignore this subsection.

\subsection{Should I use \texttt{\textbackslash sloppy}?}
\label{sec:verw-von-textttt}

Frankly speaking, the \Macro{sloppy} switch should not be used at all.
Most notably you shouldn't use it in the preamble of a document. If
line breaks appear in single paragraphs you should

\begin{enumerate}
\item check whether the right hyphenation patterns, \emph{e.g.},
  \Paket{(n)german}, and T1 fonts have been loaded (see
  \Doku{De-TeX-FAQ}\cite[section 5.3]{faq:02}), or the \Doku{UK
    FAQ}\cite[`Hyphenation', section Q.7]{ukfaq:99};
\item put your text in other words. You do not necessarily need to
  change the sentence the line break problem appears in. It may
  suffice to change the preceding, or the next sentence;
\item slightly change some parameters \TeX{} uses for line-breaking,
  and page-breaking. Axel Reichert suggested the following
  solution\footnote{Of course you may change these values according to
    taste, but beware of fiddling with \TB\texttt{emergencystretch}.
    Otherwise you'll get quite sloppy justified text as you would get
    with a rather well-known text processor.} on
  \News{de.comp.text.tex}:\footnote{The posting may be found as
    \MID{a84us0$plqcm$7@ID-30533.news.dfncis.de}{a84us0\$plqcm\$7@ID-30533.news.dfncis.de}}:

  \begin{minipage}[t]{\linewidth}
\begin{verbatim}
\tolerance 1414
\hbadness 1414
\emergencystretch 1.5em
\hfuzz 0.3pt
\widowpenalty=10000
\vfuzz \hfuzz
\raggedbottom
\end{verbatim}
    % \hspace{\baselineskip} % Yup, I know this isn't correct! ;-)
  \end{minipage}
  \par

  Note that warnings appearing with the above settings \emph{really}
  should be taken seriously. You \emph{should} consider putting your
  text in other words, then.
\end{enumerate}
%
Only if this fails you may try to typeset the following paragraph more
`loosely' using the \texttt{sloppypar} environment.
\begin{figure}[htp]
  % If someone should wonder about the additional \fussy or the
  % missing \sloppy :  Use the Source Luke!  \parbox by default typesets
  % \sloppy, so for the original linebreak \fussy is required.
  \begin{minipage}[t]{.45\textwidth}
    \centering {\fontsize{10pt}{12pt}\fontencoding{OT1}\selectfont
      \fbox{\parbox{16.27em}{\fussy%
          tatata tatata tatata tatata tatata tatata tata\-tata tatata tatata
          tatata tatata tatata tatata tata\-tata tatata tatata tatata tatata
          ta\-tatatatt\-ta tatata tatata tatata tatata tatata tatata
          ta\-ta\-ta\-ta}}}
    \caption{\LaTeX' s default settings}%
    \label{fig:beispiel-mit-latex}%
  \end{minipage}%
  \hfill%
  \begin{minipage}[t]{.45\textwidth}
    \centering    
    {\fontsize{10pt}{12pt}\fontencoding{OT1}\selectfont
      \fbox{\parbox{16.27em}{%
          tatata tatata tatata tatata tatata tatata tata\-tata tatata tatata
          tatata tatata tatata tatata tata\-tata tatata tatata tatata tatata
          ta\-tatatatt\-ta tatata tatata tatata tatata tatata tatata
          ta\-ta\-ta\-ta}}}
    \caption{This demonstrates the effect of \texttt{\string\sloppy}}
    \label{fig:beisp-mit-textttstr}
  \end{minipage}
\end{figure}

In figures~\ref{fig:beispiel-mit-latex}
and~\vref{fig:beisp-mit-textttstr} I have tried to show the effect of
\Macro{sloppy}. This also depends on the font employed. When using
Times the negative effects of \Macro{sloppy} do not show as extremely
as with, say, Computer Modern.  The effect in principle, however,
should become clear.

In \News{comp.text.tex} Markus Kohm has posted an example that shows
this effect even better. With his kind permission I quote his code
appendix~\vref{sec:beispiel-zu-sloppy}.

%----------------------- section "Obsolete" ------------------------
\section{Some obsolete commands and packages}
\label{sec:obsoletes}

Markus Kohm has written a Perl script you can test your files online
with for the most common mistakes. See
\url{http://kohm.de.tf/markus/texidate.html}.  Please note, however,
that this script is not a complete \TeX{} parser. This is why it will
only check for the most common mistakes. Please test your file first,
then post for help to a newsgroup, or to a mailing list.

\subsection{Commands}
\label{sec:befehle}

\subsubsection{Changing font style}
\label{sec:ander-des-schr}
Table~\vref{tab:befehle-zur-anderung} shows obsolete and `proper'
commands in \LaTeXe{} side by side for changing font style. Macros
called `local' only apply to their own argument whereas those called
`global/switch' will apply to all following text till the end of the
document. 

\paragraph{Why not use obsolete commands?}
\label{sec:warum-sollte-man}

Obsolete commands do not support \LaTeXe's new font selection scheme,
or NFSS. \GMacro{bf}{foo}, for example, resets all font attributes
which had been set earlier before it prints \emph{foo} in bold face.
This is why you cannot simply define a bold-italics style by
\GMacro{it}{\Macro{bf Test}} only. (This definition will produce:
{\it\bf Test}.) On the other hand, the new commands
\LMacro{textbf}{\LMacro{textit}{Test}} will behave as expected
producing: \textbf{\textit{Test}}. Apart from that, with the former
commands there is no `italic correction', cf. for instance {\it
  half}hearted (\GMacro{it}{half}\texttt{hearted}) to
\textit{half}hearted (\LMacro{textit}{half}\texttt{hearted}).

For an overview of NFSS see \cite{fntguide}.

\begin{table}
  \begin{minipage}{\textwidth}
    \renewcommand{\footnoterule}{}
    \centering
    \caption{Commands for changing font style}
    \label{tab:befehle-zur-anderung}
  \begin{tabular}{@{}lll@{}}
    \toprule
    \multicolumn{1}{@{}N}{obsolete}&
    \multicolumn{2}{C@{}}{Replacement in \LaTeXe}\\
    \cmidrule(l){2-3}
    &
    \multicolumn{1}{N}{local} &
    \multicolumn{1}{N@{}}{global/switch}\\
    \cmidrule(r){1-1}\cmidrule(lr){2-2}\cmidrule(l){3-3}
    \GMacro{bf}{\dots} & \LMacro{textbf}{\dots} & \Macro{bfseries}\\
    ---  & \LMacro{emph}{\dots}   & \Macro{em}\footnote{May be useful
      when defining macros. In continuous text \LMacro{emph}{\dots}
      should be preferred to \Macro{em}.}\\ 
    \GMacro{it}{\dots} & \LMacro{textit}{\dots} & \Macro{itshape}\\
    ---  & \LMacro{textmd}{\dots} & \Macro{mdseries}\\
    \GMacro{rm}{\dots} & \LMacro{textrm}{\dots} & \Macro{rmfamily}\\
    \GMacro{sc}{\dots} & \LMacro{textsc}{\dots} & \Macro{scshape}\\
    \GMacro{sf}{\dots} & \LMacro{textsf}{\dots} & \Macro{sffamily}\\
    \GMacro{sl}{\dots} & \LMacro{textsl}{\dots} & \Macro{slshape}\\
    \GMacro{tt}{\dots} & \LMacro{texttt}{\dots} & \Macro{ttfamily}\\
    --- & \LMacro{textup}{\dots} & \Macro{upshape}\\
    \bottomrule
  \end{tabular}
\end{minipage}
\end{table}

\subsubsection{Mathematical fractions (\texttt{\textbackslash over} vs. \texttt{\textbackslash frac})}
\label{sec:textb-over-vs}

Avoid the \Macro{over} command. \Macro{over} is a \TeX{} command which
due to the syntax differing from \LaTeX's is even more complicated to
parse or which cannot be parsed at all. The \Paket{amsmath} package
redefines \verb+\frac{}{}+ which will result in error messages when
using \Macro{over}. Another point in favour of using \verb+\frac{}{}+
is that it is easier to fill in both the fraction's numerator and
denominator, especially with more complex fractions.

\Ersetze{\texttt{\$a \textbackslash over b\$}}%
{\texttt{\$\textbackslash frac\{a\}\{b\}\$}}

\subsubsection{Centering text using \texttt{\textbackslash centerline}}
\label{sec:centerline}

The \Macro{centerline} command is another \TeX{} command you should
not use. On the one hand \Macro{centerline} is incompatible with some
\LaTeX{} packages, such as \Paket{color}. On the other hand the
package may yield unexpected results.  
\emph{E.g.}:
\begin{center}
  \begin{minipage}[t]{.45\textwidth}
\begin{verbatim}
\begin{enumerate}
\item \centerline{An item}
\end{enumerate}
\end{verbatim}
  \end{minipage}%
  \hfill%
%  \hspace{.05\textwidth}%
 \fbox{ \begin{minipage}[t]{.45\textwidth}
    \begin{enumerate}
    \item \centerline{An item}
    \end{enumerate}
  \end{minipage}}
\end{center}
\Ersetze{\LMacro{centerline}{\dots}}%
{\parbox[t]{.3\textwidth}{%
  \GMacro{centering}{\dots}\\
\textcolor{black}{or}\\
\Env{center}{\dots}
}}

\paragraph{Note:}
\label{sec:anmerkung-5}

On how to center graphics and tables see
section~\vref{sec:gleit-figure-table}.

\subsection{Class files and packages}
\label{sec:pakete}

\subsubsection{\Klasse{scrlettr} vs. \Klasse{scrlttr2}}
\label{sec:paketscrl-vs.-pakets}

\Klasse{scrlettr} class from the \KOMAScript\ bundle is obsolete. It
was replaced by \Klasse{scrlttr2}. In order to produce a
layout \emph{similar} to the former \KOMAScript\ letter class use 
class option \Option{KOMAold} which provides a compatibility mode.

\sbox{\Breite}{\texttt{\textbackslash documentclass\{scrltter\}}}
\Ersetzx[\wd\Breite]{\texttt{\textbackslash documentclass\{scrlettr\}}}%
{\texttt{\textbackslash documentclass[KOMAold]\{scrlttr2\}}}

\paragraph{Note:}
\label{sec:anmerkung-3}

For new templates and letters use the new interface. It is definitely
more flexible.

It is not possible to elaborate on the differences between the two
user interfaces in this overview. See \Doku{scrguien}~\cite{kohm:03}
for details.

\subsubsection{\Paket{epsf}, \Paket{psfig}, \Paket{epsfig} vs.
  \Paket{graphics}, \Paket{graphicx}}
\label{sec:grafikeinbindung}

The \Paket{epsf} and the \Paket{psfig} packages have been replaced by
\Paket{graphics} and \Paket{graphicx}.  \Paket{epsfig} is just a
wrapper\footnote{A `wrapper' here denotes a style file which itself
  loads another one or more style files, hence modelling functions.}
for processing old documents which had been done using \Paket{psfig}
with the \Paket{graphicx} package.

As \Paket{epsfig} uses \Paket{graphicx} internally \Paket{epsfig}
still \emph{may} be used. You should not use it, though, for new
documents.  \Paket{graphics} or \Paket{graphicx} should be preferred,
then.  \Paket{epsfig} is mainly provided for reasons of compatibility,
as mentioned above.

For the differences between \Paket{graphics}, and \Paket{graphicx} see
\Doku{grfguide}~\cite{graphicx:99}. For hints on centering graphics see
section~\vref{sec:gleit-figure-table}.

\sbox{\Breite}{\LMacro{psfig}{file=Bild,\dots}}
\Ersetze{\parbox[t]{\wd\Breite}{\Use{psfig}\\%
    \LMacro{psfig}{file=image,\dots}}}%
{\parbox[t]{.45\textwidth}{%
    \Use{graphicx}\\%
    \texttt{\textbackslash includegraphics[\dots]\{image\}}}}

\subsubsection{\Paket{doublespace} vs. \Paket{setspace}}
\label{sec:zeilenabstande}

For changing inter-line space use the \Paket{setspace} package.
\Paket{doublespace} is obsolete. It was replaced by \Paket{setspace}.
Cf. section~\vref{sec:ander-des-zeil}.

\Ersetze{\Use{doublespace}}{\Use{setspace}}

\subsubsection{\Paket{fancyheadings}, \Paket{scrpage} vs. \Paket{fancyhdr}, \Paket{scrpage2}}
\label{sec:lebende-kolumn}

The \Paket{fancyheadings} package was replaced by \Paket{fancyhdr}.
Another way to modify headings is provided by the \Paket{scrpage2}
package from the \KOMAScript\ bundle. Do not use \Paket{scrpage} for
it is obsolete. For documentation on \Paket{scrpage2} see
\Doku{scrguien}~\cite{kohm:03}.

\Ersetze{\Use{fancyheadings}}{\Use{fancyhdr}}
\Ersetze{\Use{scrpage}}{\Use{scrpage2}}

\subsubsection{The \Paket{caption} family of packages}
\label{sec:die-caption-famile}

The \Paket{caption2} package should no longer be used because there is
a new version (v3.x) of \Paket{caption}.  Please make sure to use the
latest version of this package by loading \Paket{caption} like this:
\Ersetze{\Use{caption}}{\UseV{caption}{2004/07/16}}

In case you used \Paket{caption2} before, please have a look into the
package documentation \Doku{anleitung}~\cite[section~8]{caption:04}.

\subsubsection{\Paket{isolatin}, \Paket{umlaut} vs. \Paket{inputenc}}
\label{sec:eingabe-von-umlauten}

\paragraph{Some general notes: }
\label{sec:generelles}

Basically there are four ways to input German \emph{umlauts} and
other non-ASCII characters:

\begin{enumerate}
\item \verb+H{\"u}lle+: This will work on any given system anytime.
  
  The main disadvantages, however, are that
  kerning\footnote{`Kerning' means including positive or negative
    space between characters depending on which characters are to be
    typeset.} between letters is disturbed badly; it is extremely
  complicated to input at least in a German-language text; and it is
  rather hard to read in source code.
  
  So this variant should \emph{always} be avoided due to the problems
  as far as kerning is concerned.
  
\item With \verb+H\"ulle+ or \verb+H\"{u}lle+ the aforementioned
  problems as far as kerning is concerned do not appear. It can be
  used on every system, too.
  
  However it is just as tricky to input and to read the text as with
  the above variant.
  
  This variant does make sense, however, when defining macros or style
  files for it does not require a particular text file encoding nor
  any additional packages.\label{item:die-eingabe-der}
  
\item With \Paket{(n)german} or the \Option{(n)german} option in
  \Paket{babel} German \emph{umlauts} can be input more easily
  (\verb+H"ulle+). Again this will work on all systems. As both
  \Paket{babel} and \Paket{(n)german} are available on all \TeX\
  systems there should be no problems as far as compatibility is
  concerned.
  
  However, this again is tricky to input, and the source is
  comparatively hard to read.

  This variant is best for use in continuous text. But it should be
  avoided in macro definitions and in preambles.
  
\item Direct input (\verb+H�lle+). The advantage of this variant is
  obvious. You can input and read the continuous source text just as
  any other `normal' text.
  
  On the other hand you have to tell \LaTeX{} which input encoding is
  used. There may also be problems when exchanging files between
  different systems. This is \emph{not} a problem for \TeX, or \LaTeX{} 
  itself, but it may cause \emph{problems in displaying text in
    editors} on different systems. For example, a Euro currency
  symbol encoded in iso-8859-15 (latin9) may be \emph{displayed} in an
  editor on a windows box (CP1252) as \textcurrency\ .

  This variant is quite good for continuous text. It should, however,
  be avoided in macro definitions and in preambles.
\end{enumerate}
%
To sum it up, in macros, in preambles, and in style files
\verb+H\"ulle+, or \verb+H\"{u}lle+ should be used, while in the rest
of the text you should either use \verb+H"ulle+, or \verb+H�lle+.

\paragraph{Input Encoding}
\label{sec:eingabekodierung-1}

Do \emph{not} use the packages \Paket{isolatin1}, \Paket{isolatin}, or
\Paket{umlaut} for setting input encoding! Those packages are either
obsolete, or they are not available on any given system.

Use \Paket{inputenc}. There are four options available:
\begin{description}
\item[latin1/latin9] for Unix-like systems (latin1 also works on MS
  Windows and Mac OS\,X)
\item[ansinew] for MS Windows
\item[applemac] for the Macintosh\footnote{latin1 encoding is
    recommended for OS\,X users, too, as it is better fit for
    exchanging files cross-platform than applemac. If you do so you
    should, however, check the encoding settings of your editor first.
    In the long run you might like to switch to unicode, but please
    note that unicode support in \Paket{inputenc} still is a work in
    progress at this point of time. Some users say they are content
    with \Paket{ucs} from the \textsf{unicode} package.}
\item[cp850] for OS/2
\end{description}

\Ersetze{\Use{isolatin1}}{\UseO{latin1}{inputenc}}
\Ersetze{\Use{umlaut}}{\UseO{latin1}{inputenc}}

\subsubsection{\Paket{t1enc} vs. \Paket{fontenc}}
\label{sec:schriftkodierung}

Generally speaking, the topic has been dealt with sufficiently in both
the \Doku{De-TeX-FAQ}~\cite[questions 5.3.2, 5.3.3, 10.1.10]{faq:02},
and the \Doku{UK FAQ}~\cite[`Why use \emph{fontenc} rather than
\emph{t1enc}', question 358]{ukfaq:99}. So all that remains to be said
is that \Paket{t1enc} is obsolete and hence should be replaced by
\Paket{fontenc}.  

\Ersetze{\Use{t1enc}}{\UseO{T1}{fontenc}}

\subsubsection{\Bst{natdin} vs. \Bst{dinat}}
\label{sec:liter-nach-din}

Style file \Bst{natdin} was replaced by \Bst{dinat}.
\Ersetze{\Bib{natdin}}{\Bib{dinat}}

\subsection{Fonts}
\label{sec:schriften}

`Fonts and \LaTeX' is a troublesome topic. Most discussions in
\News{de.comp.text.tex} start with the question why fonts display so
`fuzzy' in Adobe Acrobat\textsuperscript{\textregistered} Reader. Most
answers point to the \Paket{times} or \Paket{pslatex} packages.
However, those packages use completely different sets of fonts.

For an overview of \LaTeXe's New Font Selection Scheme, or NFSS see
\cite{fntguide}.

For making Computer Modern fonts display just fine in \emph{acroread}
see \Doku{De-TeX-FAQ}~\cite[question 9.2.3]{faq:02}, or \Doku{UK
  FAQ}~\cite[`The wrong type of fonts in PDF', question 114]{ukfaq:99}.

\subsubsection{\Paket{times}}
\label{sec:pakettimes}

\Paket{times} is obsolete (see \Doku{psnfss2e} \cite{psnfss:02}). It
does set \Macro{rmdefault} to Times, \Macro{sfdefault} to Helvetica,
and \Macro{ttdefault} to Courier. But it does \emph{not} use the
corresponding mathematical fonts. What's more, Helvetica is not scaled
correctly which makes it appear too big in comparison. So if you want
to use the combination Times/\/Helvetica/\/Courier you should use:

\sbox{\Breite}{\UseO{scaled=.95}{helvet}} 
\Ersetze{\Use{times}}%
{\parbox[t]{\wd\Breite}{%
    \Use{mathptmx}\\
    \UseO{scaled=.90}{helvet}\\
    \Use{courier}}}

\paragraph{Note.}
\label{sec:anmerkung-1}

The scaling factor for \Paket{helvet} together with Times should be
somewhere between $0.90$ and $0.92$.

\subsubsection{\Paket{mathptm}}
\label{sec:mathptm}

\Paket{mathptm} is the predecessor to \Paket{mathptmx}. So please use
the latter for typesetting mathematical formulae in Times.

\Ersetze{\Use{mathptm}}{\Use{mathptmx}}

\subsubsection{\Paket{pslatex}}
\label{sec:paketpslatex}

\Paket{pslatex} internally works like \Paket{mathptm}$+$
\Paket{helvet} (scaled). However, it uses a Courier font scaled too
narrowly. The main disadvantage in using \Paket{pslatex} is that it
does \emph{not} work with T1 and TS1 encodings.

\Ersetze{\Use{pslatex}}{\parbox[t]{\wd\Breite}{\Use{mathptmx}\\\UseO{scaled=.90}{helvet}\\\Use{courier}}}

\paragraph{Note on Courier for all combinations of Times/Helvetica}
\label{sec:anmerkung-zu-allen}

You do not have to load \Paket{courier} at all. You may use the usual
\texttt{cmtt} font for typewriter faces.

\subsubsection{\Paket{palatino}}
\label{sec:paketpalatino}

\Paket{palatino} behaves like \Paket{times}\,---\,apart from setting
\Macro{rmdefault} to Palatino, of course. \Paket{palatino} is
obsolete, too. This is why it should not be used any more.

\Ersetze{\Use{palatino}}{\parbox[t]{\wd\Breite}{\Use{mathpazo}\\\UseO{scaled=.95}{helvet}\\\Use{courier}}}

\paragraph{Note:}
\label{sec:anmerkung-2}

Scaling factor for \Paket{helvet} in combination with Palatino should
be set to $0.95$.

Helvetica is \emph{not} the `best' sans-serif font at all for use with
Palatino. It rather is the best \emph{freely-available} one.  He that
possesses a CorelDraw\textsuperscript{\textregistered}-CD (this may
well be an older version) can use Palatino quite well along with
Frutiger\footnote{Bitstream Humanist 777, bfr}, or
Optima\footnote{Bitstream Zapf Humanist, bop}. Walter Schmidt supplies
adaptations for using some PostScript fonts with \TeX{} on his
homepage.\footnote{Fonts for \TeX:
  \url{http://home.vr-web.de/was/fonts}} 

\subsubsection{\Paket{mathpple}}
\label{sec:paketmathpple}

This package was a predecessor to \Paket{mathpazo}. It lacks some
symbol fonts. So those fonts are taken from the Euler fonts instead.
Some other symbols are not fit for use with Palatino as the font
metrics are not correct. For details cf.
\Doku{psnfss2e}~\cite{psnfss:02}.

\subsubsection{Typesetting upright greek letters}
\label{sec:aufr-griech-buchst}

The passages I have marked as red in the following are not obsolete in
the sense of `you should not use this any more', but now editing text
is made much easier by \Paket{upgreek}. For some more hints on usage
please see the documentation \Doku{upgreek}~\cite{upgreek:01}.

\minisec{The \Paket{pifont} tricks}\nopagebreak[4]
\sbox{\Breite}{\NewCom{uppi}{\LMacro{Pisymbol}{psy}\{112\}}}
\Ersetzx[\wd\Breite]{%
  \Use{pifont}\\
  \NewCom{uppi}{\LMacro{Pisymbol}{psy}\{112\}}\\
  \Macro{uppi}\\
  \textcolor{black}{or}\\
  \NewCom[{[1]}]{upgreek}{\%\\
    ~\Macro{usefont}\{U\}\{psy\}\{m\}\{n\}\#1}\\
  \LMacro{upgreek}{p}
}%
{%
  \Use{upgreek}\\
  \$\Macro{uppi}\$
}

\minisec{The \Paket{babel} trick}

\Ersetzx[\wd\Breite]{%
  \UseO{greek,\dots}{babel}\\
  \NewCom[{[1]}]{upgreek}{\%\\
    ~\LMacro{foreignlanguage}{greek}\{\#1\}}\\
  \LMacro{upgreek}{p}
  }
  {%
  \Use{upgreek}\\
  \$\Macro{uppi}\$
}

\subsubsection{\Paket{euler} vs. \Paket{eulervm}}
\label{sec:euler-eulervm}

Use \Paket{eulervm} instead of \Paket{euler} for mathematical
typesetting.  \Paket{eulervm} is a \LaTeX{} package for using the
eulervm fonts. These are virtual math fonts based on both the Euler
and the CM fonts. consuming less of \TeX's resources and supplying
some improved math symbols. Improved \texttt{$\backslash$hslash} and
\texttt{$\backslash$hbar} are also supplied. Please see the package
documentation \Doku{eulervm}~\cite{eulervm:04} for details.

\Ersetze{\Use{euler}}{\parbox[t]{\wd\Breite}{\Use{eulervm}}}

%-------------------- section "Miscellanous" ---------------------
\section{Miscellaneous}
\label{sec:verschiedenes}

This section\,---\,apart from~\vref{sec:der-anhang}\,---\,gives some more
general advice than the `deadly sins' section, pp.\,\pageref{sec:todsunden}\,ff.

\subsection{Floats\,---\,`figure', `table'}
\label{sec:gleit-figure-table}

For centering a float environment we recommend you use
\Macro{centering} instead of \newline\LMacro{begin}{center} \dots{}
\LMacro{end}{center} because the latter will include an additional
vertical skip you can do without in most cases.
\sbox{\Breite}{\LMacro{includegraphics}{bild}}
\Ersetze{\parbox[t]{\wd\Breite}{%
    \Env{figure}{\Env{center}{\LMacro{includegraphics}{bild}}} }}%
{\parbox[t]{\wd\Breite}{%
    \Env{figure}{\Macro{centering}\\%
      \LMacro{includegraphics}{bild}} }}

\paragraph{Note:}
\label{sec:anmerkung-4}

However, when centering a region within continuous text or within a
\verb+titlepage+ environment this additional space may be welcome!

\subsection{The appendix}
\label{sec:der-anhang}

The appendix is introduced by the \Macro{appendix} \emph{command}.
Note that this is \emph{not an environment}.

\sbox{\Breite}{\LMacro{begin}{appendix}}
\Ersetze{\parbox[t]{\wd\Breite}{%
    \Env{appendix}%
    {\LMacro{section}{Blub}}}}%
{\parbox[t]{.33\textwidth}{%
    \Macro{appendix}\\
    \LMacro{section}{Blub} }}

\subsection{Mathematical typesetting}
\label{sec:mathematiksatz}

Generally speaking, you should use \Paket{amsmath} for advanced
mathematical typesetting, providing a number of new environments replacing
\texttt{eqnarray} in the first place. The main advantages of the
package are these:

\begin{itemize}
\item Spacing within and around environments is more consistent.
\item Equation numbering will be placed in a way so that they will not
  be printed over any more.
\item Some new environments, \emph{e.g.}, \texttt{split}, provide a
  solution to split up long equations easily.
\item It is easy to define new operators (similar to \Macro{sin},
  etc.)\ with proper spacing.
\end{itemize}

\paragraph{Warning:}
\label{sec:warnung}

When using \Paket{amsmath} you should \emph{never} use the
\texttt{displaymath}, \texttt{eqnarray}, or \texttt{eqnarray*}
environments because those are not supported by \Paket{amsmath}.
Otherwise this would lead to inconsistent spacing.

\texttt{\textbackslash [\dots\textbackslash]} is adapted correctly by
\Paket{amsmath}. So it may be used instead of \texttt{displaymath}.
\texttt{eqnarray}, and \texttt{eqnarray*} may be replaced by
\texttt{align}, or \texttt{align*}. For a complete overview of
\Paket{amsmath} see \Doku{amsldoc}~\cite{amsldoc:99}.

\sbox{\Breite}{\LMacro{includegraphics}{bild}}
\Ersetze{%
  \parbox[t]{\wd\Breite}{%
    \Env{eqnarray}{%
       a \&=\& b \textbackslash\textbackslash\\
       b \&=\& c \textbackslash\textbackslash\\
       a \&=\& c 
    }
  }}%
  {\parbox[t]{\wd\Breite}{%
      \Env{align}{%
        a \&= b \textbackslash\textbackslash\\
        b \&= c \textbackslash\textbackslash\\
        a \&= c
      }
    }}


\subsection{How to use \texttt{\TB graphicspath}}
\label{sec:die-verwendung-von}

There are several reasons why you should avoid the
\Macro{graphicspath} macro. Replace it by setting environment variable
\texttt{TEXINPUTS}:\footnote{Cf. David Carlisle's answer on Markus
  Kohm's `Bug-Report' at\newline
  \url{http://www.latex-project.org/cgi-bin/ltxbugs2html?pr=latex/2618}}
\begin{enumerate}
\item There are \emph{different} separators in path names on different
  platforms. While MS Windows and Unices both use a slash `/', a colon
  `:' was used on Macintosh systems before Mac OS~X.
\item \TeX{} search takes longer than with using the kpathsea library
  (with today's fast chips this is not as important an argument as it
  used to be).
\item \TeX's memory is limited, and every picture uses part of this
  memory. What's more, memory is not cleared during the compiling
  process.
\end{enumerate}
%
In a Bourne shell use 
\begin{verbatim}
$ TEXINPUTS=PictureDir:$TEXINPUTS latex datei.tex
\end{verbatim}
or add to \verb+~/.profile+ 
\begin{verbatim}
export TEXINPUTS=./PictureDir:$TEXINPUTS
\end{verbatim}%$
In the latter case the files in  \texttt{PictureDir} will be found
within the current working directory.

\noindent
Up to MS~Windows~98 the environment variable is set by adding
\begin{verbatim}
set TEXINPUTS=.\PictureDir;%TEXINPUTS%
\end{verbatim}
to your \texttt{autoexec.bat}.  On MS Windows~NT-based systems
according to the `Microsoft Knowledge Base' the variable can be set by
rightclicking at \textsf{My Computer $\rightarrow$ System Properties
  $\rightarrow$ Advanced $\rightarrow$ Environment
  variables}.\footnote{On Windows~2000 you may use:
  \textsf{Start} $\rightarrow$ Settings $\rightarrow$ Control Panel
  $\rightarrow$ System.}

The above are only some suggestions on how to proceed. I am well aware
that \texttt{TEXINPUTS} may be set in different ways. Please see the
documentation of your operating system, or of your \TeX{} distribution
for more.

\subsection{Language-specific macros\,---\,\texttt{\textbackslash*name}}
\label{sec:die-macroname-makros}

From time to time the question comes up in \News{de.comp.text.tex} how
to modify, \emph{e.g.}, the `References' heading to `Literaturliste'
or to something else. So I have compiled those macros in
table~\vref{tab:von-paketng-btw}. They have been taken from the
\Paket{german} package. Users who want to adapt macro output to other
languages may as well refer to this table as an example.

So if you want to change the heading `List of Figures' to, say,
`Pictures' you may use the following command:
\begin{verbatim}
\renewcommand*{\listfigurename}{Pictures}
\end{verbatim}
The other macros are changed in the same way respectively. With
\Paket{babel} use the \Macro{addto} macro. For more details see the
\Doku{De-TeX-FAQ}~\cite{faq:02}.

\begin{table}
  \begin{minipage}{\textwidth}
    \centering
    \caption{Macros defined by \Paket{(n)german} or by \Paket{babel}
      with the \Option{(n)german} option}
    \label{tab:von-paketng-btw}
    \begin{tabular}{@{}lll@{}}
      \toprule
      \multicolumn{1}{@{}N}{Name of macro}        & 
      \multicolumn{1}{N}{Original definition} & 
      \multicolumn{1}{N@{}}{Usual output in German}                                           \\
      \cmidrule(r){1-1}\cmidrule(lr){2-2}\cmidrule(l){3-3}
      \Macro{prefacename}                     & Preface         & Vorwort               \\
      \Macro{refname}\footnote{In \texttt{article} class only.}
                                              & References      & Literatur             \\
      \Macro{abstractname}                    & Abstract        & Zusammenfassung       \\
      \Macro{bibname}\footnote{In \texttt{report} and
      \texttt{book} classes only.}
                                              & Bibliography    & Literaturverzeichnis  \\
      \Macro{chaptername}                     & Chapter         & Kapitel               \\
      \Macro{appendixname}                    & Appendix        & Anhang                \\
      \Macro{contentsname}                    & Contents        & Inhaltsverzeichnis    \\
      \Macro{listfigurename}                  & List of Figures & Abbildungsverzeichnis \\
      \Macro{listtablename}                   & List of Tables  & Tabellenverzeichnis   \\
      \Macro{indexname}                       & Index           & Index                 \\
      \Macro{figurename}                      & Figure          & Abbildung             \\
      \Macro{tablename}                       & Table           & Tabelle               \\
      \Macro{partname}                        & Part            & Teil                  \\
      \Macro{enclname}                        & encl            & Anlage(n)             \\
      \Macro{ccname}                          & cc              & Verteiler             \\
      \Macro{headtoname}                      & To              & An                    \\
      \Macro{pagename}                        & Page            & Seite                 \\
      \Macro{seename}                         & see             & siehe                 \\
      \Macro{alsoname}                        & see also        & siehe auch            \\
      \bottomrule
    \end{tabular}
  \end{minipage}
\end{table}
\clearpage
\appendix

\begin{thebibliography}{99}\addcontentsline{toc}{section}{References}%

\bibitem{amsldoc:99} \textsc{American Mathematical Society}:
  \emph{User's Guide for the {\texttt{amsmath}} Package}. December 1999,
  Version~2.0. \biburl{CTAN://macros/latex/required/amslatex/}.
  
\bibitem{graphicx:99} \textsc{David~P. Carlisle}: \emph{Packages in the
    `graphics' bundle}. January 1999.
  \biburl{CTAN://macros/latex/required/graphics/}.

\bibitem{ukfaq:99}\textsc{Robin Fairbairns}: \emph{The UK \TeX\ FAQ.
    Your 407 Questions Answered.}  WWW, Version~3.16, 30 June 2006,
  \biburl{http://www.tex.ac.uk/faq}.
 
\bibitem{kohm:03} \textsc{Markus Kohm}, \textsc{Frank Neukam} und
  \textsc{Axel Kielhorn}: \emph{The KOMA-Script Bundle}.
  \Doku{scrguien}.
  \biburl{CTAN://macros/latex/supported/koma-script/}.

\bibitem{clsguide} \textsc{The \LaTeX3 Project}: \LaTeXe{} for
  class and package writers. 1999.
  \biburl{CTAN://macros/latex/doc/clsguide.pdf}

\bibitem{fntguide} \textsc{The \LaTeX3 Project}: \LaTeXe{} font
  selection. 2000.
  \biburl{CTAN://macros/latex/doc/fntguide.pdf}

\bibitem{usrguide} \textsc{The \LaTeX3 Project}: \LaTeXe{} for
  authors. 2001.
  \biburl{CTAN://macros/latex/doc/usrguide.pdf}

\bibitem{faq:02} \textsc{Bernd Raichle}, \textsc{Rolf Niepraschk} und
  \textsc{Thomas Hafner}: \emph{Fragen und Antworten (FAQ) �ber das
    Textsatzsystem {\TeX\ }und DANTE, Deutschsprachige
    Anwendervereinigung {\TeX\ }e.V.} WWW, Version~72. September 2003,
  \biburl{http://www.dante.de/faq/de-tex-faq/}.
  
\bibitem{upgreek:01} \textsc{Walter Schmidt}: \emph{The
    {\textsf{upgreek}} package for {\LaTeXe}}.  May 2001, Version~1.0.
  \biburl{CTAN://macros/latex/contrib/supported/was/}.
  
\bibitem{psnfss:02} \textsc{Walter Schmidt}: \emph{Using common
    PostScript fonts with {\LaTeX}}.  April 2002, PSNFSS version 9.0.
  \biburl{CTAN://macros/latex/required/psnfss/psnfss2e.pdf}

\bibitem{eulervm:04} \textsc{Walter Schmidt}: \emph{The Euler Virtual Math
    Fonts for use with {\LaTeX}}. Januar 2004, Version~3.0a.
  \biburl{CTAN://fonts/eulervm/}  

\bibitem{l2kurz:99} \textsc{Walter Schmidt}, \textsc{J�rg Knappen},
  \textsc{Hubert Partl} und \textsc{Irene Hyna}:
  \emph{{\LaTeXe}-Kurzbeschreibung}. April 1999, Version~2.1.
  \biburl{CTAN://info/lshort/german/}. English Translation available
  at \biburl{CTAN://info/lshort/english/}

\bibitem{caption:04} \textsc{Axel Sommerfeld}:
  \emph{Setzen von Abbildungs- und Tabellenbeschriftungen mit dem
    caption-Paket}. Juli 2004, Version~3.0c.
  \biburl{CTAN://macros/latex/contrib/caption/}.
\end{thebibliography}
\begin{center}$\ast$\ $\ast$\ $\ast$\end{center}
\bigskip
%---------------------------------------------------------------------
\section{An example illustrating the effect of the \texttt{\TB sloppy}
  command}
\label{sec:beispiel-zu-sloppy}
This is the example Markus Kohm published earlier:
\begin{bspcode}{\MID{8557097.gEimXdBtjU@ID-107054.user.dfncis.de}{8557097.gEimXdBtjU@ID-107054.user.dfncis.de}}%
\footnotesize
\begin{verbatim}
\documentclass{article}

\setlength{\textwidth}{20em}
\setlength{\parindent}{0pt}
\begin{document}
\typeout{First without \string\sloppy\space and underfull \string\hbox}

tatata tatata tatata tatata tatata tatata ta\-ta\-tata
tatata tatata tatata tatata tatata tatata tata\-tata
tatata tatata tatata tatata ta\-tatatatt\-ta
tatata tatata tatata tatata tatata tatata ta\-ta\-ta\-ta

\typeout{done.}

\sloppy
\typeout{Second with \string\sloppy\space and underfull \string\hbox}

tatata tatata tatata tatata tatata tatata ta\-ta\-tata
tatata tatata tatata tatata tatata tatata tata\-tata
tatata tatata tatata tatata ta\-tatatatt\-ta
tatata tatata tatata tatata tatata tatata ta\-ta\-ta\-ta

\typeout{done.}
\end{document}
\end{verbatim}
\end{bspcode}

\clearpage
%---------------------------------------------------------------------
\setlength{\columnsep}{2em}
\setlength{\columnseprule}{.5pt}
\begin{multicols}{2}
\section{GNU Free Documentation Licence}
\label{sec:fdl}

\tiny

\begin{center}

       Version 1.2, November 2002


 Copyright \copyright 2000,2001,2002  Free Software Foundation, Inc.
 
 \bigskip
 
     59 Temple Place, Suite 330, Boston, MA  02111-1307  USA
  
 \bigskip
 
 Everyone is permitted to copy and distribute verbatim copies
 of this license document, but changing it is not allowed.
\end{center}

\subsubsection*{Preamble}

The purpose of this License is to make a manual, textbook, or other
functional and useful document "free" in the sense of freedom: to
assure everyone the effective freedom to copy and redistribute it,
with or without modifying it, either commercially or noncommercially.
Secondarily, this License preserves for the author and publisher a way
to get credit for their work, while not being considered responsible
for modifications made by others.

This License is a kind of "copyleft", which means that derivative
works of the document must themselves be free in the same sense.  It
complements the GNU General Public License, which is a copyleft
license designed for free software.

We have designed this License in order to use it for manuals for free
software, because free software needs free documentation: a free
program should come with manuals providing the same freedoms that the
software does.  But this License is not limited to software manuals;
it can be used for any textual work, regardless of subject matter or
whether it is published as a printed book.  We recommend this License
principally for works whose purpose is instruction or reference.

\subsubsection*{1. APPLICABILITY AND DEFINITIONS}

This License applies to any manual or other work, in any medium, that
contains a notice placed by the copyright holder saying it can be
distributed under the terms of this License.  Such a notice grants a
world-wide, royalty-free license, unlimited in duration, to use that
work under the conditions stated herein.  The \textbf{"Document"}, below,
refers to any such manual or work.  Any member of the public is a
licensee, and is addressed as \textbf{"you"}.  You accept the license if you
copy, modify or distribute the work in a way requiring permission
under copyright law.

A \textbf{"Modified Version"} of the Document means any work containing the
Document or a portion of it, either copied verbatim, or with
modifications and/or translated into another language.

A \textbf{"Secondary Section"} is a named appendix or a front-matter section of
the Document that deals exclusively with the relationship of the
publishers or authors of the Document to the Document's overall subject
(or to related matters) and contains nothing that could fall directly
within that overall subject.  (Thus, if the Document is in part a
textbook of mathematics, a Secondary Section may not explain any
mathematics.)  The relationship could be a matter of historical
connection with the subject or with related matters, or of legal,
commercial, philosophical, ethical or political position regarding
them.

The \textbf{"Invariant Sections"} are certain Secondary Sections whose titles
are designated, as being those of Invariant Sections, in the notice
that says that the Document is released under this License.  If a
section does not fit the above definition of Secondary then it is not
allowed to be designated as Invariant.  The Document may contain zero
Invariant Sections.  If the Document does not identify any Invariant
Sections then there are none.

The \textbf{"Cover Texts"} are certain short passages of text that are listed,
as Front-Cover Texts or Back-Cover Texts, in the notice that says that
the Document is released under this License.  A Front-Cover Text may
be at most 5 words, and a Back-Cover Text may be at most 25 words.

A \textbf{"Transparent"} copy of the Document means a machine-readable copy,
represented in a format whose specification is available to the
general public, that is suitable for revising the document
straightforwardly with generic text editors or (for images composed of
pixels) generic paint programs or (for drawings) some widely available
drawing editor, and that is suitable for input to text formatters or
for automatic translation to a variety of formats suitable for input
to text formatters.  A copy made in an otherwise Transparent file
format whose markup, or absence of markup, has been arranged to thwart
or discourage subsequent modification by readers is not Transparent.
An image format is not Transparent if used for any substantial amount
of text.  A copy that is not "Transparent" is called \textbf{"Opaque"}.

Examples of suitable formats for Transparent copies include plain
ASCII without markup, Texinfo input format, \LaTeX{} input format, SGML
or XML using a publicly available DTD, and standard-conforming simple
HTML, PostScript or PDF designed for human modification.  Examples of
transparent image formats include PNG, XCF and JPG.  Opaque formats
include proprietary formats that can be read and edited only by
proprietary word processors, SGML or XML for which the DTD and/or
processing tools are not generally available, and the
machine-generated HTML, PostScript or PDF produced by some word
processors for output purposes only.

The \textbf{"Title Page"} means, for a printed book, the title page itself,
plus such following pages as are needed to hold, legibly, the material
this License requires to appear in the title page.  For works in
formats which do not have any title page as such, "Title Page" means
the text near the most prominent appearance of the work's title,
preceding the beginning of the body of the text.

A section \textbf{"Entitled XYZ"} means a named subunit of the Document whose
title either is precisely XYZ or contains XYZ in parentheses following
text that translates XYZ in another language.  (Here XYZ stands for a
specific section name mentioned below, such as \textbf{"Acknowledgements"},
\textbf{"Dedications"}, \textbf{"Endorsements"}, or \textbf{"History"}.)  
To \textbf{"Preserve the Title"}
of such a section when you modify the Document means that it remains a
section "Entitled XYZ" according to this definition.

The Document may include Warranty Disclaimers next to the notice which
states that this License applies to the Document.  These Warranty
Disclaimers are considered to be included by reference in this
License, but only as regards disclaiming warranties: any other
implication that these Warranty Disclaimers may have is void and has
no effect on the meaning of this License.

\subsubsection*{2. VERBATIM COPYING}

You may copy and distribute the Document in any medium, either
commercially or noncommercially, provided that this License, the
copyright notices, and the license notice saying this License applies
to the Document are reproduced in all copies, and that you add no other
conditions whatsoever to those of this License.  You may not use
technical measures to obstruct or control the reading or further
copying of the copies you make or distribute.  However, you may accept
compensation in exchange for copies.  If you distribute a large enough
number of copies you must also follow the conditions in section 3.

You may also lend copies, under the same conditions stated above, and
you may publicly display copies.

\subsubsection*{3. COPYING IN QUANTITY}

If you publish printed copies (or copies in media that commonly have
printed covers) of the Document, numbering more than 100, and the
Document's license notice requires Cover Texts, you must enclose the
copies in covers that carry, clearly and legibly, all these Cover
Texts: Front-Cover Texts on the front cover, and Back-Cover Texts on
the back cover.  Both covers must also clearly and legibly identify
you as the publisher of these copies.  The front cover must present
the full title with all words of the title equally prominent and
visible.  You may add other material on the covers in addition.
Copying with changes limited to the covers, as long as they preserve
the title of the Document and satisfy these conditions, can be treated
as verbatim copying in other respects.

If the required texts for either cover are too voluminous to fit
legibly, you should put the first ones listed (as many as fit
reasonably) on the actual cover, and continue the rest onto adjacent
pages.

If you publish or distribute Opaque copies of the Document numbering
more than 100, you must either include a machine-readable Transparent
copy along with each Opaque copy, or state in or with each Opaque copy
a computer-network location from which the general network-using
public has access to download using public-standard network protocols
a complete Transparent copy of the Document, free of added material.
If you use the latter option, you must take reasonably prudent steps,
when you begin distribution of Opaque copies in quantity, to ensure
that this Transparent copy will remain thus accessible at the stated
location until at least one year after the last time you distribute an
Opaque copy (directly or through your agents or retailers) of that
edition to the public.

It is requested, but not required, that you contact the authors of the
Document well before redistributing any large number of copies, to give
them a chance to provide you with an updated version of the Document.

\subsubsection*{4. MODIFICATIONS}

You may copy and distribute a Modified Version of the Document under
the conditions of sections 2 and 3 above, provided that you release
the Modified Version under precisely this License, with the Modified
Version filling the role of the Document, thus licensing distribution
and modification of the Modified Version to whoever possesses a copy
of it.  In addition, you must do these things in the Modified Version:

\begin{itemize}
\item[A.] 
   Use in the Title Page (and on the covers, if any) a title distinct
   from that of the Document, and from those of previous versions
   (which should, if there were any, be listed in the History section
   of the Document).  You may use the same title as a previous version
   if the original publisher of that version gives permission.
   
\item[B.]
   List on the Title Page, as authors, one or more persons or entities
   responsible for authorship of the modifications in the Modified
   Version, together with at least five of the principal authors of the
   Document (all of its principal authors, if it has fewer than five),
   unless they release you from this requirement.
   
\item[C.]
   State on the Title page the name of the publisher of the
   Modified Version, as the publisher.
   
\item[D.]
   Preserve all the copyright notices of the Document.
   
\item[E.]
   Add an appropriate copyright notice for your modifications
   adjacent to the other copyright notices.
   
\item[F.]
   Include, immediately after the copyright notices, a license notice
   giving the public permission to use the Modified Version under the
   terms of this License, in the form shown in the Addendum below.
   
\item[G.]
   Preserve in that license notice the full lists of Invariant Sections
   and required Cover Texts given in the Document's license notice.
   
\item[H.]
   Include an unaltered copy of this License.
   
\item[I.]
   Preserve the section Entitled "History", Preserve its Title, and add
   to it an item stating at least the title, year, new authors, and
   publisher of the Modified Version as given on the Title Page.  If
   there is no section Entitled "History" in the Document, create one
   stating the title, year, authors, and publisher of the Document as
   given on its Title Page, then add an item describing the Modified
   Version as stated in the previous sentence.
   
\item[J.]
   Preserve the network location, if any, given in the Document for
   public access to a Transparent copy of the Document, and likewise
   the network locations given in the Document for previous versions
   it was based on.  These may be placed in the "History" section.
   You may omit a network location for a work that was published at
   least four years before the Document itself, or if the original
   publisher of the version it refers to gives permission.
   
\item[K.]
   For any section Entitled "Acknowledgements" or "Dedications",
   Preserve the Title of the section, and preserve in the section all
   the substance and tone of each of the contributor acknowledgements
   and/or dedications given therein.
   
\item[L.]
   Preserve all the Invariant Sections of the Document,
   unaltered in their text and in their titles.  Section numbers
   or the equivalent are not considered part of the section titles.
   
\item[M.]
   Delete any section Entitled "Endorsements".  Such a section
   may not be included in the Modified Version.
   
\item[N.]
   Do not retitle any existing section to be Entitled "Endorsements"
   or to conflict in title with any Invariant Section.
   
\item[O.]
   Preserve any Warranty Disclaimers.
\end{itemize}

If the Modified Version includes new front-matter sections or
appendices that qualify as Secondary Sections and contain no material
copied from the Document, you may at your option designate some or all
of these sections as invariant.  To do this, add their titles to the
list of Invariant Sections in the Modified Version's license notice.
These titles must be distinct from any other section titles.

You may add a section Entitled "Endorsements", provided it contains
nothing but endorsements of your Modified Version by various
parties\,---\,for example, statements of peer review or that the text
has been approved by an organization as the authoritative definition
of a standard.

You may add a passage of up to five words as a Front-Cover Text, and a
passage of up to 25 words as a Back-Cover Text, to the end of the list
of Cover Texts in the Modified Version.  Only one passage of
Front-Cover Text and one of Back-Cover Text may be added by (or
through arrangements made by) any one entity.  If the Document already
includes a cover text for the same cover, previously added by you or
by arrangement made by the same entity you are acting on behalf of,
you may not add another; but you may replace the old one, on explicit
permission from the previous publisher that added the old one.

The author(s) and publisher(s) of the Document do not by this License
give permission to use their names for publicity for or to assert or
imply endorsement of any Modified Version.

\subsubsection*{5. COMBINING DOCUMENTS}

You may combine the Document with other documents released under this
License, under the terms defined in section 4 above for modified
versions, provided that you include in the combination all of the
Invariant Sections of all of the original documents, unmodified, and
list them all as Invariant Sections of your combined work in its
license notice, and that you preserve all their Warranty Disclaimers.

The combined work need only contain one copy of this License, and
multiple identical Invariant Sections may be replaced with a single
copy.  If there are multiple Invariant Sections with the same name but
different contents, make the title of each such section unique by
adding at the end of it, in parentheses, the name of the original
author or publisher of that section if known, or else a unique number.
Make the same adjustment to the section titles in the list of
Invariant Sections in the license notice of the combined work.

In the combination, you must combine any sections Entitled "History"
in the various original documents, forming one section Entitled
"History"; likewise combine any sections Entitled "Acknowledgements",
and any sections Entitled "Dedications".  You must delete all sections
Entitled "Endorsements".

\subsubsection*{6. COLLECTIONS OF DOCUMENTS}

You may make a collection consisting of the Document and other documents
released under this License, and replace the individual copies of this
License in the various documents with a single copy that is included in
the collection, provided that you follow the rules of this License for
verbatim copying of each of the documents in all other respects.

You may extract a single document from such a collection, and distribute
it individually under this License, provided you insert a copy of this
License into the extracted document, and follow this License in all
other respects regarding verbatim copying of that document.

\subsubsection*{7. AGGREGATION WITH INDEPENDENT WORKS}

A compilation of the Document or its derivatives with other separate
and independent documents or works, in or on a volume of a storage or
distribution medium, is called an "aggregate" if the copyright
resulting from the compilation is not used to limit the legal rights
of the compilation's users beyond what the individual works permit.
When the Document is included in an aggregate, this License does not
apply to the other works in the aggregate which are not themselves
derivative works of the Document.

If the Cover Text requirement of section 3 is applicable to these
copies of the Document, then if the Document is less than one half of
the entire aggregate, the Document's Cover Texts may be placed on
covers that bracket the Document within the aggregate, or the
electronic equivalent of covers if the Document is in electronic form.
Otherwise they must appear on printed covers that bracket the whole
aggregate.

\subsubsection*{8. TRANSLATION}

Translation is considered a kind of modification, so you may
distribute translations of the Document under the terms of section 4.
Replacing Invariant Sections with translations requires special
permission from their copyright holders, but you may include
translations of some or all Invariant Sections in addition to the
original versions of these Invariant Sections.  You may include a
translation of this License, and all the license notices in the
Document, and any Warranty Disclaimers, provided that you also include
the original English version of this License and the original versions
of those notices and disclaimers.  In case of a disagreement between
the translation and the original version of this License or a notice
or disclaimer, the original version will prevail.

If a section in the Document is Entitled "Acknowledgements",
"Dedications", or "History", the requirement (section 4) to Preserve
its Title (section 1) will typically require changing the actual
title.

\subsubsection*{9. TERMINATION}

You may not copy, modify, sublicense, or distribute the Document except
as expressly provided for under this License.  Any other attempt to
copy, modify, sublicense or distribute the Document is void, and will
automatically terminate your rights under this License.  However,
parties who have received copies, or rights, from you under this
License will not have their licenses terminated so long as such
parties remain in full compliance.

\subsubsection*{10. FUTURE REVISIONS OF THIS LICENSE}

The Free Software Foundation may publish new, revised versions
of the GNU Free Documentation License from time to time.  Such new
versions will be similar in spirit to the present version, but may
differ in detail to address new problems or concerns.  See
http://www.gnu.org/copyleft/.

Each version of the License is given a distinguishing version number.
If the Document specifies that a particular numbered version of this
License "or any later version" applies to it, you have the option of
following the terms and conditions either of that specified version or
of any later version that has been published (not as a draft) by the
Free Software Foundation.  If the Document does not specify a version
number of this License, you may choose any version ever published (not
as a draft) by the Free Software Foundation.

\subsubsection*{ADDENDUM: How to use this License for your documents}

To use this License in a document you have written, include a copy of
the License in the document and put the following copyright and
license notices just after the title page:

\begin{quote}
    Copyright \copyright  YEAR  YOUR NAME.
    Permission is granted to copy, distribute and/or modify this document
    under the terms of the GNU Free Documentation License, Version 1.2
    or any later version published by the Free Software Foundation;
    with no Invariant Sections, no Front-Cover Texts, and no Back-Cover Texts.
    A copy of the license is included in the section entitled "GNU
    Free Documentation License".
\end{quote}
    
If you have Invariant Sections, Front-Cover Texts and Back-Cover Texts,
replace the "with...Texts." line with this:

\begin{quote}
    with the Invariant Sections being LIST THEIR TITLES, with the
    Front-Cover Texts being LIST, and with the Back-Cover Texts being LIST.
\end{quote}
    
If you have Invariant Sections without Cover Texts, or some other
combination of the three, merge those two alternatives to suit the
situation.

If your document contains nontrivial examples of program code, we
recommend releasing these examples in parallel under your choice of
free software license, such as the GNU General Public License,
to permit their use in free software.

\end{multicols}
%--------------------------------------------------------------------
\section{\Doku{l2tabuen} revision history}

\begin{description}
\item[v1.8.5.7] Some minor changes by \emph{Gernot Hassenpflug} to
  English typography and style.
\item[v1.8.5.6] Adapted to the UK \TeX{} FAQ v3.16.
\item[v1.8.5.5] Minor fix in the section on \Macro{graphicspath} as to
  the correct notation regarding Mac OS~X. Thanks to \emph{Stephen
    Eglen}. Adapted to the UK \TeX{} FAQ v3.15a.
\item[v1.8.5.4] Minor additions and changes in source code applied.
  Additions to the section on \texttt{\textbackslash baselinestretch}.
  Thanks to \emph{Karl Berry}.
\item[v1.8.5.3] Adapted to the UK \TeX{} FAQ v3.13b. \emph{Legal
    Notice} modified again, hopefully for the benefit of Debian users
  and maintainers (\emph{'There are no invariant sections in this
    document.'}).
\item[v1.8.5.2] Legal notice modified.
\item[v1.8.5.1] Some typos fixed in the section on
  `graphicspath'. Thanks to \emph{Jos\'{e} Carlos Santos}.
\item[v1.8.5] Adapted to the UK \TeX{} FAQ v3.13.
\item[v1.8.4] GNU FDL made applicable to \Doku{l2tabuen}. `Revision
  history' added.
\end{description}
\begin{center}$\ast$\ $\ast$\ $\ast$\end{center}
\end{document}
%%% Local variables:
%%% mode: LaTeX
%%% TeX-master: t
%%% coding: iso-8859-1
%%% End:
